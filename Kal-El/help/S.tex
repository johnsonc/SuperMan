% Created 2013-10-22 Tue 07:24
\documentclass[11pt]{article}
\usepackage[utf8]{inputenc}
\usepackage{graphicx}
\usepackage{longtable}
\usepackage{float}
\usepackage{wrapfig}
\usepackage{amsmath}
\usepackage{hyperref}
\usepackage{color}
\usepackage{listings}
\author{TagTeam, KKHolst}
\date{\today}
\title{emacs SuperMan: prolog}
\hypersetup{
  pdfkeywords={},
  pdfsubject={},
  pdfcreator={Emacs 24.3.1 (Org mode 8.0.7)}}
\begin{document}

\maketitle
\section*{Getting started with your new SuperMan(er)}
\label{sec-1}

\begin{description}
\item[{S:}] SuperMan is a project which manages all your other projects.
\item[{Y:}] This sound useful. But what exactly is considered a project?
\item[{S:}] A project is an entry in a file, where the file-name is
defined by the emacs variable `superman-home'. 
An entry looks like this:
\end{description}

\begin{verbatim}
**** Extensions of the Hodges-Lehmann estimator for structural equation models
    :PROPERTIES:
    :NICKNAME: HodgesLehmann
    :LOCATION: ~/research/Methods/
    :CATEGORY: Work
    :OTHERS: Terre Antivirus, Audrey Blanche
    :LastVisit: <2013-05-31 Fri 19:59>
    :END:
\end{verbatim}

The properties have the following meaning:

\begin{verbatim}
**** Description text
    :PROPERTIES:
    :NICKNAME: Short project name which is also used
               as directory name below LOCATION 
    :LOCATION: A directory on your computer in which
               the main project files are saved
    :CATEGORY: The project category
    :OTHERS: Names of your collaborators
             of the project
    :LastVisit: Automatically updated time-string
                indicating when you last looked into
                this project
    :END:
\end{verbatim}

\begin{description}
\item[{Y:}] Ok. How do I register my projects.
\item[{S:}] That is easy: first open the SuperMan via M-x S RET,
then type N for superman-new-project. You will be
prompted for NICKNAME and CATEGORY.
\item[{Y:}] hmm, I will give it a try \ldots{} last question: Does the SuperMan project contain itself?
\item[{S:}] Nice question. Probably the answer is: \emph{Yes! But,\ldots{}}. To find a better answer please read logicomix (www.logicomix.com).
\end{description}
% Emacs 24.3.1 (Org mode 8.0.7)
\end{document}