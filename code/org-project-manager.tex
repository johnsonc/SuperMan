% Created 2012-10-07 Sun 15:03
\documentclass[11pt]{article}
\usepackage[utf8]{inputenc}
\usepackage[T1]{fontenc}
\usepackage{fixltx2e}
\usepackage{graphicx}
\usepackage{longtable}
\usepackage{float}
\usepackage{wrapfig}
\usepackage{soul}
\usepackage{textcomp}
\usepackage{marvosym}
\usepackage{wasysym}
\usepackage{latexsym}
\usepackage{amssymb}
\usepackage{hyperref}
\tolerance=1000
\usepackage{color}
\usepackage{listings}
\usepackage{authblk}
\usepackage{natbib}
\usepackage[T1]{fontenc}
\renewcommand*\familydefault{\sfdefault}
\usepackage[table,usenames,dvipsnames]{xcolor}
\definecolor{lightGray}{gray}{0.98}
\definecolor{medioGray}{gray}{0.83}
\rowcolors{1}{medioGray}{lightGray}
\usepackage{attachfile}
\usepackage{array}
\author{Thomas Alexander Gerds}
\affil{Department of Biostatistics, University of Copenhagen, Denmark}
\author{Klaus K\"ahler Holst}
\affil{Department of Biostatistics, University of Copenhagen, Denmark}
\author{Jochen Knaus}
\affil{Department of Medical Biometrie and Medical Informatics, University of Freiburg, Freiburg, Germany}
\newcommand{\sfootnote}[1]{\renewcommand{\thefootnote}{\fnsymbol{footnote}}\footnote{#1}\setcounter{footnote}{0}\renewcommand{\thefootnote}{\arabic{foot note}}}
\makeatletter\def\blfootnote{\xdef\@thefnmark{}\@footnotetext}\makeatother
\usepackage{color}
\lstset{
keywordstyle=\color{blue},
commentstyle=\color{red},
stringstyle=\color[rgb]{0,.5,0},
basicstyle=\ttfamily\small,
columns=fullflexible,
breaklines=true,        % sets automatic line breaking
breakatwhitespace=false,    % sets if automatic breaks should only happen at whitespace
numbers=left,
numberstyle=\ttfamily\tiny\color{gray},
stepnumber=1,
numbersep=10pt,
backgroundcolor=\color{white},
tabsize=4,
showspaces=false,
showstringspaces=false,
xleftmargin=.23in,
frame=single,
basewidth={0.5em,0.4em}
}
\providecommand{\alert}[1]{\textbf{#1}}

\title{An emacs-org project manager for applied statisticians}
%\author{Thomas Alexander Gerds}
\date{\today}
\hypersetup{
  pdfkeywords={},
  pdfsubject={},
  pdfcreator={Emacs Org-mode version 7.9.2}}

\begin{document}

\maketitle

\section{Introduction}
\label{sec-1}

  
  This document simulaneously describes and defines a project manager
  for applied statistical workflows based on the fabulous and popular
  emacs \texttt{org-mode}. Really it describes and automates one out of many
  possibilities to customize \texttt{org-mode}.
  
\subsection{Getting started}
\label{sec-1-1}

   
   The first thing to do is to define some new projects. Obviously, you
   would rather read in your existing projects. But stay cool and read on
   until this is explained below.
   
\subsubsection{The project manager file}
\label{sec-1-1-1}

    
    
\begin{itemize}

\item What is a project?\\
\label{sec-1-1-1-1}%
As in real life an org-pro project can be many different things. The
     prototype project consists of a \emph{location}, that is a directory on
     your computer, and an index file. (Actually, only one of \emph{index} and
     \emph{location} is required.)
     

\item How to define a project?\\
\label{sec-1-1-1-2}%
Projects are defined in the file \emph{org-pro-file}
     (yes-you-need-to-create-this-file) in the following format.
     

\lstset{language=org}
\begin{lstlisting}
* Cat 1
** Subcat 1
*** Subsubcat 1
**** Learning org (My first org-pro project)
:PROPERTIES:
:NICKNAME: howto-org
:INDEX:  ~/knowhow/howto.org
:LOCATION: ~/knowhow/
:OTHERS: justme
:END:     
**** Test org-pro
:PROPERTIES:
:NICKNAME: howto-org
:INDEX:  ~/knowhow/howto.org
:LOCATION: ~/knowhow/
:OTHERS: Alphonse Quack
:END:
\end{lstlisting}

\end{itemize} % ends low level
\subsection{Git support}
\label{sec-1-2}
\subsubsection{Requirements}
\label{sec-1-2-1}


You need to install the program git.
\subsubsection{Git hub}
\label{sec-1-2-2}


To do push changes to a github repository it is useful to set up git  
\href{https://help.github.com/articles/set-up-git}{https://help.github.com/articles/set-up-git}
and to be free of typing the username and password when doing ``git push''.

If you use the SSH repo URL instead, SSH keys are used for
authentication. This guide offers help generating and using an SSH key
pair:  \href{https://help.github.com/articles/generating-ssh-keys}{https://help.github.com/articles/generating-ssh-keys}

\end{document}
